% CSC 300: Professional Responsibilities
% Dr. Clark Turner

% Two Column Format
\documentclass[11pt]{article}
%this allows us to specify sections to be single or multi column so that things
% like title page and table of contents are single column
\usepackage{multicol}

\usepackage{setspace}
\usepackage{url}

%%% PAGE DIMENSIONS
\usepackage{geometry} % to change the page dimensions
\geometry{letterpaper}

\begin{document}

\title{\vfill Maintenance Courses in a Software Engineering Curriculum} %\vfill gives us the black space at the top of the page
\author{
James Pearson \vspace{10pt} \\
CSC 300: Professional Responsibilities  \vspace{10pt} \\
Section 1 \vspace{10pt} \\
Dr. Clark Turner \vspace{10pt} \\
}
%\date{October 22, 2010} %Or use \Today for today's Date
\date{\today}

\maketitle

\vfill  %in combinaion with \newpage this forces the abstract to the bottom of the page
\begin{abstract}
Although software maintenance makes up most of the work a software engineer will do in the workplace, Cal Poly requires Software Engineering undergraduates to take only two courses that cover software maintenance.  Considering this, is it ethical for the Computer Science department to state that the program is designed to produce software professionals?  The ACM's Software Engineering Code of Ethics lists maintenance as one of the primary focuses for a Software Engineer; given this, the department should require a more thorough instruction in the art of software maintenance for Software Engineering students.
\end{abstract}

\thispagestyle{empty} %remove page number from title page
\newpage


%Create a table of contents with all headings of level 3 and above.
%http://en.wikibooks.org/wiki/LaTeX/Document_Structure#Table_of_contents has
%info on customizing the table of contents
\thispagestyle{empty}  %Remove page number from TOC
\tableofcontents

\newpage

%end the 1 column format


%start 2 column format
\begin{multicols}{2}
%Start numbering first page of content as page 1
\setcounter{page}{1}
%%%%%%%%%%%%%%%%%%%%
%%% Known Facts  %%%
%%%%%%%%%%%%%%%%%%%%
\section{Facts}

The Cal Poly catalog states that the Software Engineering undergraduate degree "prepares students to become software professionals who develop software 
products on time, within budget, and that meet customer requirements". \cite{catalogDept}  It also says that the program differs from the similar Computer Science undergraduate degree program in four ways, one of which is having "classes [that] include significant learning in engineering and management areas such as quality assurance, testing, metrics, maintenance, configuration management and interpersonal management skills." \cite{catalogDept}

Three courses in the Computer Science department's section of the 2011-13 Cal Poly Catalog have the word "maintenance" in their course description. \cite{catalogCourses}  In one of these, CSC 358 Computer System Administration, the reference is to "system maintenance". \cite{catalogCourses}  Given the course title and that the course is billed as teaching "fundamental concepts of Unix system administration", \cite{catalogCourses} we can conclude that the maintenance described is of an operating system and its components, rather than a single piece of software in which the maintainer is altering code.  Thus, this course is not of relevance to our discussion.

CSC 309's course description refers to "maintenance of large software systems", while CSC 406's simply mentions "software maintenance". \cite{catalogCourses}

George E. Stark claims that "maintenance consumes between 60 percent and 80 percent of a typical product's total software lifecycle expenditures". \cite{stark97}  Girish Parekh says that it is two-thirds of the lifecycle and consumes at least 50 percent of most programmers' time. \cite{parekh}

Andrews and Lutfiyya, while preparing in 1998 for a software maintenance course at the University of Western Ontario, found no courses at other universities in which students performed maintenance activities on legacy software. \cite{Andrews:2000:ERS:794188.794320} They also found that "most computer science programs offer[ed] no more than two software engineering course". \cite{Andrews:2000:ERS:794188.794320}

%%%%%%%%%%%%%%%%%%%%%%%%%
%%% Research Question %%%
%%%%%%%%%%%%%%%%%%%%%%%%%
\section{Research Question}
Given that Cal Poly requires Software Engineering undergraduates to take only two courses that cover software maintenance, is it ethical for the Computer Science department to state that the program is designed to produce software professionals?

%%%%%%%%%%%%%%%%%%%%%%%%%
%%% Extant Arguments from External Sources %%%
%%%%%%%%%%%%%%%%%%%%%%%%%
\section{Extant arguments}

Andrews and Lutfiyya found that their software maintenance course "gave students valuable experience in the qualitatively different task of software maintenance". \cite{Andrews:2000:ERS:794188.794320}  Engle, Ford and Korson state that it is "important for students to have experienced [software maintenance]". \cite{engle}

I have not found any arguments against including maintenance as a part of software engineering curriculum.

%%%%%%%%%%%%%%%%
%%% Analysis %%%
%%%%%%%%%%%%%%%%
\section{Analysis}

The ACM's Software Engineering Code explicitly mentions software maintenance as one of the things software engineers should focus on. \cite{secode}

Contrast the amount of time spent on software maintenance in the field with the amount of time devoted to it in education.  Then, discuss how the SE code states that software engineers should be educated in software maintenance.

%%%%%%%%%%%%%%%%
%%% Conclusion %%%
%%%%%%%%%%%%%%%%
\section{Conclusion}
The conclusion is a summary of your entire anal- ysis. It should reiterate the answer your audience has been forming while reading your analysis. New information should never be introduced in your conclusion. \cite{texTemp}

%end the two column format
\end{multicols}
\newpage

%cite all the references from the bibtex you haven't explicitly cited
\nocite{*}

\bibliographystyle{termpaper/IEEEannot}

\bibliography{termpaper/texreport}

\end{document}
